\documentclass[12pt]{article}
\usepackage{amsmath}
\usepackage{graphicx}
\usepackage{listings}
\usepackage{amsfonts}
\usepackage{amssymb}
\usepackage{hyperref}
\usepackage{tocloft}
\usepackage{multirow}

% \documentclass[tikz,border=10pt]{standalone}
\usepackage{tikz}
\usetikzlibrary{automata, positioning, arrows}

\usepackage[framed,autolinebreaks,useliterate]{mcode}

\title{Coursework 3 Answer}
\author{Junbiao Li - 209050796}
\date{\today}

\renewcommand{\cftsecleader}{\cftdotfill{\cftdotsep}}
% 对于 \section
\renewcommand{\cftsubsecleader}{\cftdotfill{\cftdotsep}}
% 对于 \subsection

\begin{document}

\maketitle
\tableofcontents

\clearpage
\section{Question a)}

Since we need to find that the minimum variance portfolio of the 10 stocks. We
first calculate the weekly returns of the 10 stocks and expected return and
variance of return, which can be calculate as
\[
    \begin{aligned}
        \mathbb{E}\left[R_i\right] & =\frac{1}{60} \sum_{t=1}^{60} R_i(t),
        \quad i=1, \cdots, 10                                              \\
        \sigma_i^2                 & =\frac{1}{60}
        \sum_{t=1}^{60}\left(R_i(t)-\mathbb{E}\left[R_i\right]\right)^2, \quad
        i=1,
        \cdots, 10                                                         \\
        \sigma_{i j}               & =\frac{1}{60}
        \sum_{t=1}^{60}\left(R_i(t)-\mathbb{E}\left[R_i\right]\right)\left(R_j(t)-\mathbb{E}\left[R_j\right]\right),
        \quad i \neq j
    \end{aligned}
\]
where \(R_i(t)=\frac{P_i(t)-P_i(t-1)}{P_i(t-1)}\).

Hence, using MATLAB, we have the following results:
\[
    \text{expected return} =\left(
    \begin{array}{r}
            0.0032  \\
            0.0035  \\
            -0.0005 \\
            0.0010  \\
            -0.0016 \\
            0.0180  \\
            -0.0017 \\
            0.0049  \\
            -0.0045 \\
            0.0014
        \end{array}\right),
\]

\[
    \begin{aligned}
         & \text{covariances} = \\
         & \left(
        \begin{array}{rrrrrrrrrr}
            0.0018 & 0.0002  & 0.0009 & 0.0008 & 0.0006  & 0.0009 &
            0.0009 & 0.0004  & 0.0009 & 0.0005                      \\
            0.0002 & 0.0008  & 0.0006 & 0.0003 & 0.0003  & 0.0002 &
            0.0004 & -0.0001 & 0.0004 & 0.0003                      \\
            0.0009 & 0.0006  & 0.0023 & 0.0012 & 0.0014  & 0.0012 &
            0.0013 & 0.0001  & 0.0011 & 0.0007                      \\
            0.0008 & 0.0003  & 0.0012 & 0.0020 & 0.0009  & 0.0010 &
            0.0011 & 0.0005  & 0.0010 & 0.0008                      \\
            0.0006 & 0.0003  & 0.0014 & 0.0009 & 0.0016  & 0.0007 &
            0.0011 & -0.0001 & 0.0009 & 0.0006                      \\
            0.0009 & 0.0002  & 0.0012 & 0.0010 & 0.0007  & 0.0032 &
            0.0008 & 0.0003  & 0.0001 & 0.0007                      \\
            0.0009 & 0.0004  & 0.0013 & 0.0011 & 0.0011  & 0.0008 &
            0.0016 & 0.0002  & 0.0013 & 0.0007                      \\
            0.0004 & -0.0001 & 0.0001 & 0.0005 & -0.0001 & 0.0003 &
            0.0002 & 0.0010  & 0.0003 & 0.0003                      \\
            0.0009 & 0.0004  & 0.0011 & 0.0010 & 0.0009  & 0.0001 &
            0.0013 & 0.0003  & 0.0022 & 0.0006                      \\
            0.0005 & 0.0003  & 0.0007 & 0.0008 & 0.0006  & 0.0007 &
            0.0007 & 0.0003  & 0.0006 & 0.0009                      \\
        \end{array}
        \right)
    \end{aligned}
\]

Now we can solve the following optimization problem using MATLAB with the
solver for quadratic objective functions with linear constraints.
The code is shown in Appendix \ref{lst:a}.

\[
    \begin{aligned}
        \min _{x_1, \cdots, x_n} & \quad \sigma_p^2         \\
        \text { Subject to }     & \quad \sum_{i=1}^n x_i=1 \\
    \end{aligned}
\]

Setting \verb|H=2∗covariances;f=zeros(n_stocks,1);Aeq=ones(1,n_stocks);beq=1;|
The result is

\begin{itemize}
    \item Weights of the portfolio: \\ \(
          \begin{array}{rr}
              \text{AHT.L:}  & 0.031339  \\
              \text{CCH.L:}  & 0.446839  \\
              \text{FRAS.L:} & -0.158178 \\
              \text{MNG.L:}  & -0.069499 \\
              \text{RMV.L:}  & 0.297457  \\
              \text{RR.L:}   & 0.040424  \\
              \text{SDR.L:}  & -0.034508 \\
              \text{SHEL.L:} & 0.390948  \\
              \text{STJ.L:}  & -0.008146 \\
              \text{TSCO.L:} & 0.063325  \\
          \end{array}
          \)
    \item Variance of the portfolio: 0.000314
    \item Expected return of the portfolio: 0.004029
\end{itemize}

\section{Question b)}

In this question, we need to find the Minimum variance portfolio with a
specific expected return, which is

\[\begin{aligned}
        \min _{x_1, \cdots, x_n} & \sigma_p^2                \\
        \text { Subject to }     & \mathbb{E}_p=\mathbb{E}_0 \\
                                 & \sum_{i=1}^n x_i=1
    \end{aligned}\]

Like the previous question, we can solve the problem using MATLAB, however, we
need to add the constraint \(\mathbb{E}_p=\mathbb{E}_0\). The code is shown in
Appendix \ref{lst:b}.

Setting \verb|H=2*covariances;f=zeros(n_stocks,1);target_return=0.015;|
\verb|Aeq=[ones(1,n_stocks);expected_returns];beq=[1;target_return];| The
result is

\begin{itemize}
    \item Weights of the portfolio: \\ \(
          \begin{array}{rr}
              \text{AHT.L:}  & 0.060745  \\
              \text{CCH.L:}  & 0.686850  \\
              \text{FRAS.L:} & -0.364956 \\
              \text{MNG.L:}  & -0.129785 \\
              \text{RMV.L:}  & 0.214895  \\
              \text{RR.L:}   & 0.570912  \\
              \text{SDR.L:}  & -0.231232 \\
              \text{SHEL.L:} & 0.455774  \\
              \text{STJ.L:}  & -0.026677 \\
              \text{TSCO.L:} & -0.236526 \\
          \end{array}
          \)
    \item Variance of the portfolio: 0.001171
    \item Expected return of the portfolio: 0.015000
\end{itemize}

\section{Question c)}

In this question, we need to Maximizing the utility, which is
\(u=\mathbb{E}_p-0.5 \sigma_p^2\).
Hence we need to setting
\verb|kappa=0.5;H=2*kappa*covariances;f=-expected_returns;|
\verb|Aeq=ones(1, n_stocks);beq=1;| The code is shown in Appendix \ref{lst:c}.

The result is

\begin{itemize}
    \item Weights of the portfolio: \\ \(
          \begin{array}{rr}
              \text{AHT.L:}  & 0.407439  \\
              \text{CCH.L:}  & 3.516466  \\
              \text{FRAS.L:} & -2.802765 \\
              \text{MNG.L:}  & -0.840536 \\
              \text{RMV.L:}  & -0.758477 \\
              \text{RR.L:}   & 6.825118  \\
              \text{SDR.L:}  & -2.550508 \\
              \text{SHEL.L:} & 1.220050  \\
              \text{STJ.L:}  & -0.245152 \\
              \text{TSCO.L:} & -3.771635 \\
          \end{array}
          \)
    \item Maximum utility portfolio: 0.074028
    \item Expected return of the portfolio: 0.144340
\end{itemize}

\section{Question d)}

In this question, we are required to find the maximum return portfolio with a
certain level of risk such that
\(\sigma^2_p = 0.004\) which is

\[\begin{aligned}
        \max _{x_1, \cdots, x_n} & \mathbb{E}_p       \\
        \text { Subject to }     & \sigma^2_p = 0.004 \\
                                 & \sum_{i=1}^n x_i=1
    \end{aligned}\]

Since it is not a quadratic programming problem, we need to use the Lagrangian
method to solve it.

We have the Lagrangian function

\[
    \begin{aligned}
        L(x_1, \cdots, x_n, \mu, \lambda) & =\mathbb{E}_p-\mu\left(\sum_{i=1}^n
        x_i-1\right)-\lambda\left(\sigma^2_p-0.004\right)                                   \\
                                          & =\mathbf{x}^T\mathbf{r} -\mu \left(\mathbf{x}^T
        \mathbf{1}-1\right)-\lambda
        \left(\mathbf{x}^T\mathbf{\Sigma}\mathbf{x}-0.004\right)                            \\
        \text{where } \mathbf{r}          & =\left(\mathbb{E}_1, \cdots,
        \mathbb{E}_n\right)^T                                                               \\
        \mathbf{\Sigma}                   & =\left(\sigma_{i j}\right)_{n \times n}
    \end{aligned}
\]

Now we can letting the derivatives of the Lagrangian with respect to \(x_1,
\cdots, x_n, \mu, \lambda\) be zero, which is

\[
    \begin{aligned}
        \frac{\partial L}{\partial x_i}     & =\mathbf{r} - \mu^T \mathbf{1} -
        2\lambda \mathbf{\Sigma} \mathbf{x} = 0                                \\
        \frac{\partial L}{\partial \mu}     & =\mathbf{x}^T \mathbf{1} - 1 = 0 \\
        \frac{\partial L}{\partial \lambda} & =\mathbf{x}^T \mathbf{\Sigma}
        \mathbf{x} - 0.004 = 0
    \end{aligned}
\]

Hence, we can use MATLAB to solve the equation
The code is shown in Appendix \ref{lst:d}. The result is

\begin{itemize}
    \item Weights of the portfolio: \\ \(
          \begin{array}{rr}
              \text{AHT.L:}  & 0.092301  \\
              \text{CCH.L:}  & 0.944393  \\
              \text{FRAS.L:} & -0.586839 \\
              \text{MNG.L:}  & -0.194476 \\
              \text{RMV.L:}  & 0.126301  \\
              \text{RR.L:}   & 1.140152  \\
              \text{SDR.L:}  & -0.442326 \\
              \text{SHEL.L:} & 0.525337  \\
              \text{STJ.L:}  & -0.046562 \\
              \text{TSCO.L:} & -0.558282 \\
          \end{array}
          \)
    \item Variance of the portfolio: 0.004000
    \item Expected return of the portfolio: 0.026772
\end{itemize}

\clearpage
\appendix
\section*{Appendix}
\addcontentsline{toc}{section}{Appendix}
\renewcommand{\thesubsection}{\Alph{subsection}}

\subsection{MATLAB Code}

\begin{lstlisting}[label=lst:a,
    caption=Question a)]
    clear
    filename = 'Historical Prices.xlsx';
    opts = detectImportOptions(filename);
    data = readtable(filename, opts);
    
    % calculate weekly returns
    prices = table2array(data(:, 2:end));
    returns = (prices(2:end, :) - prices(1:end-1, :)) ./ prices(1:end-1, :);
    
    % calculate expected returns and variances
    expected_returns = mean(returns);
    variances = var(returns);
    
    % calculate covariances
    covariances = cov(returns);
    
    % set up the optimization problem
    n_stocks = size(returns, 2);
    H = 2 * covariances;
    f = zeros(n_stocks, 1);
    
    Aeq = ones(1, n_stocks);
    beq = 1;
    
    options = optimoptions('quadprog', 'Display', 'off');
    [x, fval] = quadprog(H, f, [], [], Aeq, beq, [], [], [], options);
    
    optimal_weights = x;
    variance_portfolio = fval;
    expected_return_portfolio = expected_returns * optimal_weights;
    
    fprintf('Weights of the portfolio:\n');
    for i = 1:n_stocks
        fprintf('%s: %f\n', data.Properties.VariableNames{i+1}, optimal_weights(i));
    end
    fprintf('Variance of the portfolio: %f\n', variance_portfolio);
    fprintf('Expected return of the portfolio: %f\n', expected_return_portfolio);
\end{lstlisting}

\begin{lstlisting}[label=lst:b,
    caption=Question b)]
    clear
    filename = 'Historical Prices.xlsx';
    opts = detectImportOptions(filename);
    data = readtable(filename, opts);
    
    % calculate weekly returns
    prices = table2array(data(:, 2:end));
    returns = (prices(2:end, :) - prices(1:end-1, :)) ./ prices(1:end-1, :);
    
    % calculate expected returns and variances
    expected_returns = mean(returns);
    variances = var(returns);
    
    % calculate covariances
    covariances = cov(returns);
    
    % set up the optimization problem
    n_stocks = size(returns, 2);
    H = 2 * covariances;
    f = zeros(n_stocks, 1); % Since we are minimizing variance, there is no linear term
    
    target_return = 0.015;
    Aeq = [ones(1, n_stocks); expected_returns];
    beq = [1; target_return];
    
    options = optimoptions('quadprog', 'Display', 'off');
    [x, fval] = quadprog(H, f, [], [], Aeq, beq, [], [], [], options);
    
    optimal_weights = x;
    variance_portfolio = fval;
    expected_return_portfolio = expected_returns * optimal_weights;
    
    
    fprintf('Weights of the portfolio:\n');
    for i = 1:n_stocks
        fprintf('%s: %f\n', data.Properties.VariableNames{i+1}, optimal_weights(i));
    end
    fprintf('Variance of the portfolio: %f\n', variance_portfolio);
    fprintf('Expected return of the portfolio: %f\n', expected_return_portfolio);
\end{lstlisting}

\begin{lstlisting}[label=lst:c,
    caption=Question c)]
    clear
    filename = 'Historical Prices.xlsx';
    opts = detectImportOptions(filename);
    data = readtable(filename, opts);
    
    % calculate weekly returns
    prices = table2array(data(:, 2:end));
    returns = (prices(2:end, :) - prices(1:end-1, :)) ./ prices(1:end-1, :);
    
    % calculate expected returns and variances
    expected_returns = mean(returns);
    variances = var(returns);
    
    % calculate covariances
    covariances = cov(returns);
    
    kappa = 0.5;
    n_stocks = length(expected_returns);
    H = 2*kappa * covariances;
    f = -expected_returns;
    
    % linear equalities: sum(weights) = 1
    Aeq = ones(1, n_stocks);
    beq = 1;
    
    options = optimoptions('quadprog', 'Display', 'off');
    [x, fval] = quadprog(H, f, [], [], Aeq, beq, [], [], [], options);
    
    optimal_weights = x;
    utility = -fval;
    expected_return_portfolio = expected_returns * optimal_weights;
    
    fprintf('Weights of the portfolio:\n');
    for i = 1:n_stocks
        fprintf('%s: %f\n', data.Properties.VariableNames{i+1}, optimal_weights(i));
    end
    fprintf('Maximum utility portfolio: %f\n', utility);
    fprintf('Expected return of the portfolio: %f\n', expected_return_portfolio);
\end{lstlisting}

\begin{lstlisting}[label=lst:d,
    caption=Question d)]
    clear
    filename = 'Historical Prices.xlsx';
    opts = detectImportOptions(filename);
    data = readtable(filename, opts);
    
    % calculate weekly returns
    prices = table2array(data(:, 2:end));
    returns = (prices(2:end, :) - prices(1:end-1, :)) ./ prices(1:end-1, :);
    
    % calculate expected returns and variances
    expected_returns = mean(returns);
    variances = var(returns);
    
    % calculate covariances
    covariances = cov(returns);
    n_stocks = length(expected_returns);
    
    syms x1 x2 x3 x4 x5 x6 x7 x8 x9 x10 mu lamb
    weights = [x1, x2, x3, x4, x5, x6, x7, x8, x9, x10];
    
    % Define the objective function
    expected = expected_returns*weights';
    
    % Define the constraints
    g1 = sum(weights) - 1 == 0; 
    var = weights * covariances * weights'; 
    g2 = var - 0.004 == 0; 
    
    W = expected - mu * lhs(g1) - lamb * lhs(g2);
    
    % Calculate the derivatives of the Lagrangian
    dL_dx = arrayfun(@(i) diff(W, weights(i)) == 0, 1:length(weights));
    dL_dmu = diff(W, mu) == 0;
    dL_dlamb = diff(W, lamb) == 0;
    
    system = [dL_dx, dL_dmu, dL_dlamb];
    solutions = vpasolve(system, [weights, mu, lamb]);
    
    optimal_weights = [solutions.x1, solutions.x2, solutions.x3, solutions.x4, solutions.x5, solutions.x6, solutions.x7, solutions.x8, solutions.x9, solutions.x10];
    expected_return_portfolio = sum(optimal_weights .* expected_returns);
    
    variance_portfolio = optimal_weights * covariances * optimal_weights';
    
    
    fprintf('Weights of the portfolio:\n');
    for i = 1:n_stocks
        fprintf('%s: %f\n', data.Properties.VariableNames{i+1}, optimal_weights(i));
    end
    fprintf('Variance of the portfolio: %f\n', variance_portfolio);
    fprintf('Expected return of the portfolio: %f\n', expected_return_portfolio);
\end{lstlisting}
\end{document}
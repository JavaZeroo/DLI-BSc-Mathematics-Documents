\documentclass[12pt]{article}
\usepackage{amsmath}
\usepackage{graphicx}
\usepackage{amsfonts}
\usepackage{amssymb}
\usepackage{hyperref}
\usepackage{tocloft}
\usepackage{listings}
\usepackage{xcolor}
\usepackage{textcomp}
\usepackage{pdfpages}
\usepackage{ulem}

\usepackage[framed,numbered,autolinebreaks,useliterate]{mcode}

\renewcommand{\lstlistingname}{Code Block}

\title{Coursework 1}
\author{Junbiao Li - 209050796}
\date{\today}

\lstset{
    language=Matlab,
    basicstyle=\scriptsize,
    numbers=none
}

\begin{document}

% \includepdf[pages=-]{sign.pdf}

\maketitle

\section{Question 1}

\subsection{a) No-Arbitrage Condition}

First, we need to verify whether the given \( (u, d) \) pairs satisfy the
no-arbitrage condition. According to the reference material, the no-arbitrage
condition is as follows:

\[
    d < 1 + \rho < u
\]

The monthly interest rate, denoted by \(\rho_m\), can be derived from the annual interest
rate represented by \(\rho\), as follows:

\[
    \rho_m = \left(1 + \rho\right)^{1/12} - 1
\]

The calculated value is \(\rho_m \approx 0.00439\).

For all four pairs of \( (u, d) \) values, it is found that they satisfy the
no-arbitrage condition \( d < 1 + \rho_m < u \).

\begin{itemize}
    \item For \(u = 1.020, d = 0.980\): \(0.98 < 1 + 0.00439 < 1.02\) holds
          true.
    \item For \(u = 1.006, d = 0.985\): \(0.985 < 1 + 0.00439 < 1.006\)
          holds true.
    \item For \(u = 1.025, d = 0.975\): \(0.975 < 1 + 0.00439 < 1.025\)
          holds true.
    \item For \(u = 1.013, d = 0.987\): \(0.987 < 1 + 0.00439 < 1.013\)
          holds true.
\end{itemize}

Therefore, these factors all satisfy the no-arbitrage condition and can be
utilized for subsequent option pricing.

\subsection{b) Option Pricing}

\subsubsection{Calculating \( q \)-Probabilities}

For each \( (u, d) \) pair, there are corresponding risk-neutral probabilities
\( q_u \) and \( q_d \). These probabilities calculated as follows:

\[
    q_u=\frac{1+\rho_m-d}{u-d}, \quad q_d=1-q_u
\]

The calculated values are:

\begin{enumerate}
    \item For \(u = 1.020, d = 0.980\): \( q_u = 0.610, q_d = 0.390 \)
    \item For \(u = 1.006, d = 0.985\): \( q_u = 0.923, q_d = 0.077 \)
    \item For \(u = 1.025, d = 0.975\): \( q_u = 0.588, q_d = 0.412 \)
    \item For \(u = 1.013, d = 0.987\): \( q_u = 0.669, q_d = 0.331 \)
\end{enumerate}

\subsubsection{Time 0 Option Price of the European Call Option.}

Recall that:
\[
    \text { call option claim }=\left(S_T-k\right)+
\]

% Since we have different \( (u, d) \) pairs, we need to calculate the option price saparately for each path. \
% We use two for loops to iterate through all possible paths. \

After calculating the \( q \)-probabilities, we need to use these probabilities
and the \( (u, d) \) pairs to step through the option values for each path
saparately using two for loops(shown in Code \ref{lst:pathcode}).
Given a specific path that consists of a series of \( u \) and \( d \), we can
calculate the stock price at the end of the path \( S_T \) using the following
formula:

\[
    S_T = S_0 \times \prod_{i=1}^{20} Y_i
\]

Where \( Y_i \) is either \( u \) or \( d \) at the \( i \) th moment in the
path.

The value of the option under the path is then calculated using the following
formula:

\[
    \text{Option Value at } S_T = \max(S_T - K, 0)
\]

Finally, we need to weight these option values using risk-neutral probabilities
and discount them back to the \( t=0 \) time using the following formula:

\[
    V_0 = \frac{\sum \left( \text{Option Value at } S_T \times
        \text{Probability of the Path} \right)}{(1+\rho_m)^{20}}
\]

\begin{lstlisting}[label=lst:pathcode,
    caption=Use two for loops to iterate through all possible paths.,
    float]
V0_normal = 0;
% Generate all possible paths (each path is a sequence of u's and d's)
all_paths = dec2bin(0:(2^n_periods - 1), n_periods) - '0';
% Iterate through all paths to calculate their contributions to the option price
for i = 1:size(all_paths, 1)
    path = all_paths(i, :);
    S = S0;
    path_prob = 1;  % Initialize the probability of this path occurring
    % Calculate the final stock price along this path
    for t = 1:length(path)
        index = mod(t - 1, 4) + 1;  % Determine which (u, d) pair to use
        u = ud_pairs(index, 1);
        d = ud_pairs(index, 2);
        q_u = q_probabilities(index, 1);
        q_d = q_probabilities(index, 2);
        % Update the stock price and path probability
        if path(t) == 1
            S = S * u;
            path_prob = path_prob * q_u;
        else
            S = S * d;
            path_prob = path_prob * q_d;
        end
    end
    % Calculate the option value for this path
    option_value = max(S - K, 0) * path_prob;
    % Update V0 for the normal option
    V0_normal = V0_normal + option_value;
end
\end{lstlisting}

In traversing all possible paths using two loops, we calculated the option
value for each path following the steps above. By aggregating the weighted
option values of all the paths and discounting them back to \( t=0 \), \textbf{\uwave{we
obtain the \( t=0 \) time-unarbitrage-free price of the European Call Option \(
V_0 = 0.2641 \). }}

The complete MATLAB code is shown in Appendix \ref{apx:code}.

\subsection{c) Price of the Barrier Option}

Based on the calculation of the price of the regular option, the Barrier Option
can only be exercised if the stock price reaches or exceeds a given level \( L
\) at expiration. Therefore, we add an `if` judgment statement which is in Code
Block \ref{lst:bcode} to the code to
check if the stock price \( S_T \) at the end of each path is greater than or
equal to \( L \). Only in this case will the option value of the path be
included in the calculation of the barrier option price \(V_0 \) at the time of
\(t=0 \).

\textbf{\uwave{Ultimately, we can get the Price of the Barrier Option $V_0$ is found to be $0.2145$}}

\begin{lstlisting}[label=lst:bcode,
    caption=Add one if statement to check if the stock price \( S_T \) at the end of each path is greater than or equal to \( L \).,
    float]
V0_barrier = 0;
% Iterate through all paths to calculate their contributions to the option price
for i = 1:size(all_paths, 1)
    %... same as the normal option
    % Calculate the final stock price along this path
    for t = 1:length(path)
        %... same as the normal option
    end
    %... same as the normal option
    % Check if this path leads to a stock price >= L at maturity
    if S >= L
        V0_barrier = V0_barrier + option_value;
    end
end
\end{lstlisting}

\subsection{d) Price of the Lookback Put Option}

To calculate the price of this option, we must traverse all potential price
paths and assess each one. In particular, we are required to:

\begin{enumerate}
    \item Track the maximum price (\( S_{\max} \)) of the underlying asset
          along each path.
    \item Compute the price of the underlying asset at maturity (\( S_T
          \)).
    \item Use \( S_{\max} \) and \( S_T \) to calculate the option value
          for each path which is \[ \text{Option Value at } S_T = \max(S_{\max}
              - S_T, 0)\]
    \item Sum the weighted option values for all paths and discount back to
          \( t = 0 \).
\end{enumerate}

We weight these option values in the same way as regular options, using
q-probabilities, and discount back to \( t=0 \). \textbf{\uwave{In this way, we
obtain the no-arbitrage price \( V_0 = 0.1251 \) of the lookback put option at time \( t=0 \).}}

\appendix

\section{MATLAB Code}
\label{apx:code}

\begin{lstlisting}[label=lst:code,
    caption=Complete MATLAB Code]
% Define ud_pairs array
ud_pairs = [
    1.02, 0.98;
    1.006, 0.985;
    1.025, 0.975;
    1.013, 0.987
];

% Given parameters
annual_interest_rate = 0.054;  % Annual interest rate of 5.4%
S0 = 5.35;  % Initial stock price
K = 5.65;  % Strike price of the European call option
n_periods = 20;  % Number of periods in the binomial tree
L = 6;

% Calculate the monthly interest rate
monthly_interest_rate = (1 + annual_interest_rate)^(1/12) - 1;
q_probabilities = (1 + monthly_interest_rate - ud_pairs(:,2)) ./ (ud_pairs(:,1) - ud_pairs(:,2));
q_probabilities = [q_probabilities, 1 - q_probabilities];

% Initialize V0 for the normal, barrier, and lookback put options
V0_normal = 0;
V0_barrier = 0;
V0_lookback_put = 0;

% Generate all possible paths (each path is a sequence of u's and d's)
all_paths = dec2bin(0:(2^n_periods - 1), n_periods) - '0';

% Iterate through all paths to calculate their contributions to the option price
for i = 1:size(all_paths, 1)
    path = all_paths(i, :);
    S = S0;
    S_max = S0;  % Initialize the maximum stock price for this path
    path_prob = 1;  % Initialize the probability of this path occurring
    
    % Calculate the final stock price along this path
    for t = 1:length(path)
        index = mod(t - 1, 4) + 1;  % Determine which (u, d) pair to use
        u = ud_pairs(index, 1);
        d = ud_pairs(index, 2);
        q_u = q_probabilities(index, 1);
        q_d = q_probabilities(index, 2);
        
        % Update the stock price and path probability
        if path(t) == 1
            S = S * u;
            path_prob = path_prob * q_u;
        else
            S = S * d;
            path_prob = path_prob * q_d;
        end

        % Update the maximum stock price for this path
        S_max = max(S, S_max);
    end
    
    % Calculate the option value for this path
    option_value = max(S - K, 0) * path_prob;
    
    % Update V0 for the normal option
    V0_normal = V0_normal + option_value;
    
    % Check if this path leads to a stock price >= L at maturity
    if S >= L
        V0_barrier = V0_barrier + option_value;
    end

    % Calculate the lookback put option value for this path
    lookback_put_value = max(S_max - S, 0) * path_prob;
    
    % Update V0 for the lookback put option
    V0_lookback_put = V0_lookback_put + lookback_put_value;
end

% Discount the option price back to time 0
V0_normal = V0_normal / ((1 + monthly_interest_rate) ^ n_periods);
V0_barrier = V0_barrier / ((1 + monthly_interest_rate) ^ n_periods);
V0_lookback_put = V0_lookback_put / ((1 + monthly_interest_rate) ^ n_periods);

% Display the results
V0_normal
V0_barrier
V0_lookback_put

% Expected output:
% V0_normal = 0.2641
% V0_barrier = 0.2145
% V0_lookback_put = 0.1251
  
\end{lstlisting}

\end{document}
